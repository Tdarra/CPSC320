% ======================= Pre-Amble =========================
      
%Format
\documentclass[11pt, oneside]{article}   	% use "amsart" instead of "article" for AMSLaTeX format 
                     						%imports package {article} and specify option(s) [11pt, oneside]
\usepackage{geometry}                		% See geometry.pdf to learn the layout options. There are lots. 
    \geometry{letterpaper}                   		% ... or a4paper or a5paper or ... 
    %\geometry{landscape}                		% Activate for rotated page geometry

\usepackage[parfill]{parskip}    		        % Activate to begin paragraphs with an empty line rather than an indent

    %Colours
    \usepackage{graphicx, subcaption}
    \usepackage[usenames, dvipsnames]{color}     % font colour:    \textcolor{<colour>}{text}
          									%highlight text:  \colorbox{<color>}{text}
    \usepackage{soul}						%highlight text: \hl{}     %only  yellow								
    									%list of colours: https://www.sharelatex.com/learn/Using_colours_in_LaTeX
    									
    %Bullets
    \usepackage{enumerate}     %specify type of enumeration: \being{enumerate}[<type of enumeration>]
    
    %Footnote Spacing
    \setlength{\footnotesep}{0.4cm}                  %specify spacing b/w footnotes
    \setlength{\skip\footins}{0.6cm}                    % space b/w footnotes and textbody

	%Sections
%	\makeatletter
%	% we use \prefix@<level> only if it is defined
%	\renewcommand{\@seccntformat}[1]{
%	  \ifcsname prefix@#1\endcsname
%	    \csname prefix@#1\endcsname
%	  \else
%	    \csname the#1\endcsname\quad 
%	  \fi}
%	% define \prefix@section
%	\newcommand\prefix@section{Question \thesection}
%	\makeatother

%	\makeatletter
%	\def\@seccntformat#1{%
%	  \expandafter\ifx\csname c@#1\endcsname\c@section
%	  Question \thesection
%	  \else
%	  \csname the#1\endcsname\quad
%	  \fi}
%	\makeatother



%Mattematics
    %American Mathematics Society packages
    \usepackage{amsmath}	   %math
    \usepackage{amssymb}       %symbols
    \usepackage{amsthm}          %theorems
    \newtheorem{proposition}{Proposition}

    %QED
    \newcommand*{\QEDA}{\hfill\ensuremath{\blacksquare}}         %make qed filled square:    \QEDA
    \newcommand*{\QEDB}{\hfill\ensuremath{\square}}               %make qed empty square: \QEDB 
    \renewcommand\qedsymbol{\ensuremath{\blacksquare}}		%Proof environment
    
    % Proofs
	\newtheorem{thm}{Theorem}[section]
	\newtheorem{lem}[thm]{Lemma}
	\newtheorem{prop}[thm]{Proposition}
	\newtheorem*{cor}{Corollary}
	
	\theoremstyle{definition}
	\newtheorem{defn}{Definition}[section]
	\newtheorem{conj}{Conjecture}[section]
	\newtheorem{exmp}{Example}[section]
	
	\theoremstyle{remark}
	\newtheorem*{rem}{Remark}
	\newtheorem*{note}{Note}
	
	% Symbol Shortcuts
	\newcommand{\R}{\ensuremath{\mathbb{R}}}
	\newcommand{\C}{\ensuremath{\mathbb{C}}}
	\newcommand{\Z}{\ensuremath{\mathbb{Z}}}
	\newcommand{\Q}{\mathbb{Q}}
	\newcommand{\N}{\mathbb{N}}
	
	% Augmented Matrix
	\makeatletter
	\renewcommand*\env@matrix[1][*\c@MaxMatrixCols c]{%
  	\hskip -\arraycolsep
 		 \let\@ifnextchar\new@ifnextchar
 		 \array{#1}}
	\makeatother
	
    %\numberwithin{counterA}{counterB} 	% replaces counterA by counterb.countera
	%\numberwithin{equation}{question} 	% for equations: (5) -> (6.1)
    
    % Spacing Units
	\usepackage{siunitx}				% Syntax: \SI{value}{unit}
								%	-> e.g. $\SI{-9.81}{ms^{-2}}$
	
    % MATLAB in sentence (need mcode.sty in folder)
	%\usepackage[]{mcode} % http://www.mathworks.com/matlabcentral/fileexchange/8015-m-code-latex-package
	% Syntax:
	%	- In Sentence: \mcode{<code>}
	%	- Block of code: \begin{lstlisting} <code> \end{lstlisting}
	%	- Footnote: \footnote{ \mcodefn{ <code> } }
	%	- External m-file
	%		-> Entire file: \lstinputlisting{/SOME/PATH/FILENAME.M}
	%		-> Certain lines (i.e. skip header): \lstinputlisting[firstline=6, lastline=15]{/SOME/PATH/FILENAME.M}



%Figures
\usepackage{caption}
\captionsetup[figure]{labelfont=bf}    %make figure labels boldface
\captionsetup[table]{labelfont=bf}     %make table labels boldface

\usepackage[hidelinks]{hyperref}                % Allows for clickable references

    %Tables
    \usepackage[none]{hyphenat}                    % Stops breaking-up words in a table (i.e. no hyphens)                                                             
    
    \usepackage{array}   
        \newcolumntype{x}[1]{>{\centering\let\newline\\\arraybackslash\hspace{0pt}}p{#1}}       %center fixed column width: x{<len>}                      
        \newcolumntype{$}{>{\global\let\currentrowstyle\relax}}                                 % let us apply things (e.g. bold/italicize) to entire row            
        \newcolumntype{^}{>{\currentrowstyle}}
        \newcommand{\rowstyle}[1]{\gdef\currentrowstyle{#1} #1\ignorespaces}
    
    %Images
    \graphicspath{ {images/} }                          %directory that your images are located in within your current directory
    
    %Diagrams
    \usepackage[latin1]{inputenc}
    \usepackage{tikz}
    	\tikzset{line/.style={-latex'}}
        \usepackage{tkz-berge}
        \usetikzlibrary{shapes,arrows}
        \usetikzlibrary{patterns}			%Specify colours of stuff (e.g. vertices): 
        										%	-> set style: \tikzset{VertexStyle/.append style = {minimum size = 8pt, inner sep = 0pt}} 
											%	-> change individual vertices: \AddVertexColor{white}{1,2} 


%Bibliography
\usepackage[numbers,sort&compress]{natbib}   %for multiple references: sorts  (i.e. [1,2] NOT [2, 1] )
                                           				  %                                     compresses (i.e. [1-3] )
\usepackage[nottoc]{tocbibind}                            %add bibliography to table of contents


%Miscellaneous
\usepackage{dirtytalk}    %quotations: use \say  
\usepackage{wasysym} 	 % \smiley{} \frownie{} \blacksmiley{}
\usepackage{hyperref}


%================================== Header & Footer =======================================
\usepackage{fancyhdr}
\usepackage{lastpage}      %ensures you can reference LastPage (i.e. Page 2 of 10)

\renewcommand{\headrulewidth}{0.4pt}		%Decorative Header line: thickness={0.4pt}
\renewcommand{\footrulewidth}{0.4pt}		%Decorative Footer line: thickness={0.4pt}

\setlength{\headheight}{13.6pt} 		%space b/w top of page & header
\setlength{\headsep}{0.3in}		%space b/w page header and body

%Make Header & Footer    
\pagestyle{fancy}
    \lhead{S. Knill and T. Darra} 		% controls the left corner of the header
    \chead{} 					% controls the center of the header
    \rhead{} 					% controls the right corner of the header
    \lfoot{} 					% controls the left corner of the footer
    \cfoot{Page~\thepage\ of \pageref{LastPage}} 				% controls the center of the footer
    												%Page~\thepage\  if just want Page x
    \rfoot{}			 		% controls the right corner of the footer


% ================================== Document =============================================
\begin{document}

% Title Page
\title{CPSC 320 --- Assignment 2 \\
\line(1,0){360} \\              %(slope x, y){length of line}
}
\author{
Stephanie Knill (54882113) \\
Taj Darra (43350115) \\
Due: October 6, 2016}

\date{}                   % Activate:  display a given date (e.g. {August 4} ) or no date (empty {} )
                                    %No activate: display current date
\maketitle

%\thispagestyle{empty}                   %Remove header from this (first) page. Change empty -> plain to keep numbering
%								-> Doesn't matter in this case (b/c title page)
%\cleardoublepage


% ================= Questions ================

\section*{Collaboration Policy}
\emph{All group members have read and followed the guidelines for academic conduct in CPSC 320. As part of those rules, when collaborating with anyone outside my group, (1) I and my collaborators took no record but names away, and (2) after a suitable break, my group created the assignment I am submitting without help from anyone other than the course staff.}

\section{Olympic Scheduling}
You are in charge of a live-streaming YouTube channel for the Olympics
that promises never to interrupt an event. (So, once you start playing
an event, you must play only that event from the time it starts to the
time it finishes.) You have a list of the events, where each event
includes its: \textbf{start time, finish time (which must be after its start time), and expected audience value}. Your goal is to \textbf{make a schedule to broadcast the most valuable complete events}. \textbf{The best schedule is the one with the highest-valued event}; \textbf{in case of ties, compare second-highest valued events, and so on}. (So, for example, you
obviously \textbf{will} include the single highest-valued event in the
Olympics---presumably the hockey gold medal game---no matter what else
it blocks you from showing.)

(Times when you're not broadcasting events will be filled with ``human
interest stories'' that have zero value; so, they're irrelevant.)

\textbf{ASSUME: all event values are distinct and all event times are distinct.} I.e., for any two values $v_i$ and $v_j$ with $i \neq j$,
$v_i \neq v_j$. The same holds for start and end times (e.g., for any
two start times $s_i$ and $s_j$ with $i \neq j$, $s_i \neq
s_j$). Further, for any two start and finish times $s_i$ and $f_j$,
whether $i = j$ or not, $s_i \neq f_j$.
\subsection{Na\"ive Algorithm}

Consider the following algorithm. Assume that deleting an event from a
list of events takes constant time.

\begin{verbatim}
Naive(E):
  result = new empty list of events
  while E is not empty:
    bestEvent = E[0]
    for each e in E:
      if value(e) > value(bestEvent):
        bestEvent = e
    delete bestEvent from E
    for each e in E:
      if start(e) < finish(bestEvent) and finish(e) > start(bestEvent):
        delete e from E
    add bestEvent to result
  return result
\end{verbatim}
\subsubsection{Finiteness}

Briefly sketch a proof that the \texttt{while} loop in the algorithm above
terminates. You need not give a formal proof, but you should include
all key insights in the proof.
\subsubsection{Efficiency}

Give and briefly justify a good asymptotic bound on the runtime of the
algorithm.
\subsubsection{Correctness}
Briefly sketch a proof that the algorithm is correct. You need not
give a formal proof, but you should include all key insights in the
proof.
\subsection{Reduction on Simplified Problem}

To make the Olympic Broadcasting problem simpler, we completely remove
start time and finish times from the problem. So, now events only have
values (not times), and a ``schedule'' is just a set of selected
events. To make it slightly harder again, you are not allowed to
select two events $i$ and $j$ if their values are within 10 units of
each other: $|v_i - v_j| \leq 10$.

Give a correct reduction from this simplified Olympic Broadcasting
problem to the sorting problem (where you provide both a list of items
and a function to compare two items). Your reduction should take $O(n
\lg n)$ time.

\textbf{NOTE:} You will likely find that (a) you can solve this with a single
call to the sorting problem's solution algorithm and (b) producing the
sorting instance is the easier part and transforming the solution to
sorting into a solution to this simplified Olympic Broadcasting
problem is the harder part. Don't forget to do both!
\subsection{Olympic Reduction, BONUS ONLY}

This was significantly harder than we intended it to be! So, we
removed it from the quiz/assignment. It's a bonus problem worth two
CPSC 320 bonus points for extremely clear, correct, and efficient
responses. (Extremely clear reductions that take $O(n)$ time---not
counting an $O(1)$ number of calls to a sorting algorithm---may
receive 3 bonus points, but we don't know if such reductions are
possible.)

Give a correct and efficient reduction from the Olympic broadcasting
problem to the sorting problem (where you provide both a list of items
and a function to compare two items). Your reduction---combined with
an $O(n \lg n)$ sorting algorithm---should be asymptotically more
efficient than the na\"ive algorithm above.
\subsubsection{Correctness}

Briefly sketch a proof that your algorithm is correct. You need not
give a formal proof, but you should include all key insights in the
proof.
\subsubsection{Efficiency}

Give and briefly justify a good asymptotic bound on the runtime
of \textbf{just} your reduction, \textbf{not} including the call to the sorting
algorithm. So, for the purposes of this asymptotic bound, you can
imagine that we somehow solve sorting in constant time. (Note: it's
possible to give a reduction that takes $O(n)$ time.)


\end{document} 